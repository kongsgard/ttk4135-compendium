\ctitle{Tips \& Tricks}

\paragraph{LP}\index{LP}
Remember to read the task closely, and see whether it asks for the given resources to be \textit{fully utilized}. If not, then introduce slack variables in your constraints.

\paragraph{Prediction horizon}{Prediction horizon} Generally, the prediction horizon $N$ should be chosen such that
\begin{equation}
    \text{dominant dynamics} < N < \text{control interval}
\end{equation}

\paragraph{Slack variables}\index{Slack variables} We normally have tighter bounds on the state variables than reality since some of the constraints are hard and must always be satisfied. Therefore, the state constraints (A.9d) may be violated for all the time. In this case a feasible point may not exist and a control input may not be available.

Fix: (soften the constraints by using slack variables)
\begin{equation}
    \min_{z \in \mathbb{R}^n} f(z) = \cdots + p^\top \epsilon + \frac{1}{2} \epsilon ^\top S \epsilon
\end{equation}

\begin{equation}
    \begin{split}
        x^{\text{low}} < \:&x_t < x^{\text{high}}\\
        &\Downarrow\\
        x^{\text{low}} - \epsilon_t < \: &x_t < x^{\text{high}} + \epsilon_t
    \end{split}
\end{equation}

\paragraph{Linearization of a constraint}
%
Use the following formula to linearize a constraint (eg. for use in a QP problem):
\begin{equation}
    \nabla c_i(x_i)^\top p + c_1(x_i)
\end{equation}

\paragraph{Limit on control input} The constraint $-\Delta u^{\mathrm{high}} \leq \Delta u_t \leq \Delta u^{\mathrm{high}}$ is a limit on change of control input and/or limit wear and tear of control input.

\paragraph{Rosenbrock function}\index{Rosenbrock function} A nonconvex function on the form
\begin{equation}
\begin{split}
    f(x_1, x_2) &= (a - x_1)^2 +b(x_2-x_1^2)^2\\
    a, b &\geq 0
\end{split}
\end{equation}

\paragraph{State feedback infinite LQ control} Let system $(A,B)$ be stabilizable and $(A,D)$ detectable. $D$ is defined by $Q=D^\top D$. Then the closed loop system given by the optimal solution is asymptotically stable.

\paragraph{What to do if the Hessian is positive semi-definite?} (When using the SQP algorithm) The KKT matrix will be singular. Three options to account for this:
\begin{itemize}[nolistsep,noitemsep]
    \item Using Quasi-Newton method
    \item Adding a multiple of the identity matrix
    \item Modified Cholesky factorization
\end{itemize}
