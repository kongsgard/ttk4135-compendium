\ctitle{Background Material}

\index{Matrix factorizations}
\hskip-0.5cm
\begin{tabularx}{\linewidth}{X X X X}
	\multicolumn{4}{c}{\textbf{Matrix factorizations}} \\
	\textbf{Cholesky} & \textbf{LU} & \textbf{QR} & \textbf{Symmetric indefinite}\\
	\hline
	$A = LL^\top$ & $PA = LU$ & $AP = QR$ & $PAP^\top = LBL^\top$\\ %Definitions
	\hline
	& $Ax = b$ & &\\
	& $L\underbrace{(Ux)}_{y} = b$ & &\\
	& Triangular forward substitution & &\\
	& $Ly = b$ for y & &\\
	& Triangular backward substitution & &\\
	& $Ux = y$ for x & &\\
	\hline
	$A \in \mathbb{R}^{nxn}$ & $A \in \mathbb{R}^{nxn}$ & $A \in \mathbb{R}^{mxn}$ & $A=A^T$\\
	$A=A^T$ (Symmetric) & & (only real A) & A can be indefinite\\
	$A \succ 0$ & & & \\
\end{tabularx}

\cstitle{Matrix calculus}

\paragraph{Derivative}
\begin{equation}
\begin{split}
  \nabla(c^\top \mathbf{x}) &= c\\
  \nabla(\mathbf{x}^\top c) &= c
\end{split}
\end{equation}
%
\begin{equation}
  \nabla \left( \frac{1}{2} \mathbf{x}^\top G \mathbf{x} \right) = \frac{1}{2}G \mathbf{x} + \frac{1}{2}G^\top \mathbf{x}
\end{equation}

\paragraph{Gradient}
\begin{equation}
  \nabla f(\mathbf{x}) = \begin{bmatrix} \frac{\partial f}{\partial \mathbf{x}}^\top \end{bmatrix}
  = \begin{bmatrix}
  \frac{\partial f}{\partial x_1}\\
  \vdots\\
  \frac{\partial f}{\partial x_n}
  \end{bmatrix}
\end{equation}

\paragraph{Hessian}
\begin{equation}
  \nabla_{xx} f(\mathbf{x}) = \begin{bmatrix}
    \frac{\partial f}{\partial x_1^2} & \frac{\partial f}{\partial x_1 \partial x_2} & \cdots\\
    \frac{\partial f}{\partial x_2 \partial x_1} & \frac{\partial f}{\partial x_2^2} &\\
    \vdots & & \ddots
  \end{bmatrix}
\end{equation}

\paragraph{Jacobian}
\begin{equation}
  J f(\mathbf{x}) = \begin{bmatrix}
    \frac{\partial f_1}{\partial x_1} & \frac{\partial f_1}{\partial x_2} & \cdots\\
    \frac{\partial f_2}{\partial x_1} & \frac{\partial f_2}{\partial x_2} &\\
    \vdots & & \ddots
  \end{bmatrix}
\end{equation}

\cstitle{Elements of Analysis}

\paragraph{Lipschitz continuous}\index{Lipschitz continuous}
\begin{equation}
  || f(x_1) - f(x_0) || \leq L || x_1 - x_0 ||, \quad \forall \: x_0, x_1 \in \mathcal{N}
\end{equation}

\paragraph{Mean value theorem}\index{mean value theorem}
\begin{equation}
  f(x+p) = f(x) + \nabla f(x + \alpha p)^\top p
\end{equation}